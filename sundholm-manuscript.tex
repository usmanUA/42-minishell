%%%%%%%%%%%%%%%%%%%%%%%%%%%%%%%%%%%
%This is the LaTeX ARTICLE template for RSC journals
%Copyright The Royal Society of Chemistry 2016
%%%%%%%%%%%%%%%%%%%%%%%%%%%%%%%%%%%

\documentclass[twoside,twocolumn,9pt]{article}
\usepackage{extsizes}
\usepackage[super,sort&compress,comma]{natbib} 
\usepackage{threeparttable}
\usepackage[version=3]{mhchem}
\usepackage[left=1.5cm, right=1.5cm, top=1.785cm, bottom=2.0cm]{geometry}
\usepackage{balance}
\usepackage{makecell}
\usepackage{mathptmx}
\usepackage{sectsty}
\usepackage{tabularx}
\usepackage{graphicx} 
\usepackage{subfig}
\usepackage{hyperref}
\usepackage{cleveref}
\usepackage{lastpage}
\usepackage[format=plain,justification=justified,singlelinecheck=false,font={stretch=1.125,small,sf},labelfont=bf,labelsep=space]{caption}
\usepackage{float}
\usepackage{fancyhdr}
\usepackage{fnpos}
\usepackage[english]{babel}
\addto{\captionsenglish}{%
  \renewcommand{\refname}{Notes and references}
}
\usepackage{array}
\usepackage{droidsans}
\usepackage{charter}
\usepackage[T1]{fontenc}
\usepackage[usenames,dvipsnames]{xcolor}
\usepackage{setspace}
\usepackage[compact]{titlesec}
%%%Please don't disable any packages in the preamble, as this may cause the template to display incorrectly.%%%


\usepackage{epstopdf}%This line makes .eps figures into .pdf - please comment out if not required.

\definecolor{cream}{RGB}{222,217,201}

\begin{document}

\pagestyle{fancy}
\thispagestyle{plain}
\fancypagestyle{plain}{
%%%HEADER%%%
\renewcommand{\headrulewidth}{0pt}
}
%%%END OF HEADER%%%

%%%PAGE SETUP - Please do not change any commands within this section%%%
\makeFNbottom
\makeatletter
\renewcommand\LARGE{\@setfontsize\LARGE{15pt}{17}}
\renewcommand\Large{\@setfontsize\Large{12pt}{14}}
\renewcommand\large{\@setfontsize\large{10pt}{12}}
\renewcommand\footnotesize{\@setfontsize\footnotesize{7pt}{10}}
\makeatother

\renewcommand{\thefootnote}{\fnsymbol{footnote}}
\renewcommand\footnoterule{\vspace*{1pt}% 
\color{cream}\hrule width 3.5in height 0.4pt \color{black}\vspace*{5pt}} 
\setcounter{secnumdepth}{5}

\makeatletter 
\renewcommand\@biblabel[1]{#1}            
\renewcommand\@makefntext[1]% 
{\noindent\makebox[0pt][r]{\@thefnmark\,}#1}
\makeatother 
\renewcommand{\figurename}{\small{Fig.}~}
\sectionfont{\sffamily\Large}
\subsectionfont{\normalsize}
\subsubsectionfont{\bf}
\setstretch{1.125} %In particular, please do not alter this line.
\setlength{\skip\footins}{0.8cm}
\setlength{\footnotesep}{0.25cm}
\setlength{\jot}{10pt}
\titlespacing*{\section}{0pt}{4pt}{4pt}
\titlespacing*{\subsection}{0pt}{15pt}{1pt}
%%%END OF PAGE SETUP%%%

%%%FOOTER%%%
\fancyfoot{}
\fancyfoot[LO,RE]{\vspace{-7.1pt}\includegraphics[height=9pt]{head_foot/LF}}
\fancyfoot[CO]{\vspace{-7.1pt}\hspace{13.2cm}\includegraphics{head_foot/RF}}
\fancyfoot[CE]{\vspace{-7.2pt}\hspace{-14.2cm}\includegraphics{head_foot/RF}}
\fancyfoot[RO]{\footnotesize{\sffamily{1--\pageref{LastPage} ~\textbar  \hspace{2pt}\thepage}}}
\fancyfoot[LE]{\footnotesize{\sffamily{\thepage~\textbar\hspace{3.45cm} 1--\pageref{LastPage}}}}
\fancyhead{}
\renewcommand{\headrulewidth}{0pt} 
\renewcommand{\footrulewidth}{0pt}
\setlength{\arrayrulewidth}{1pt}
\setlength{\columnsep}{6.5mm}
\setlength\bibsep{1pt}
%%%END OF FOOTER%%%

%%%FIGURE SETUP - please do not change any commands within this section%%%
\makeatletter 
\newlength{\figrulesep} 
\setlength{\figrulesep}{0.5\textfloatsep} 

\newcommand{\topfigrule}{\vspace*{-1pt}% 
\noindent{\color{cream}\rule[-\figrulesep]{\columnwidth}{1.5pt}} }

\newcommand{\botfigrule}{\vspace*{-2pt}% 
\noindent{\color{cream}\rule[\figrulesep]{\columnwidth}{1.5pt}} }

\newcommand{\dblfigrule}{\vspace*{-1pt}% 
\noindent{\color{cream}\rule[-\figrulesep]{\textwidth}{1.5pt}} }

\newcolumntype{C}{>{\centering\arraybackslash}X}
\newcolumntype{R}{>{\raggedleft\arraybackslash}X}
\newcolumntype{L}{>{\raggedright\arraybackslash}X}

\newcommand*\dage[1]{{\color{red}{{DS: #1}}}}
\newcommand*\mikael[1]{{\color{blue}{{MPJ: #1}}}}
\newcommand*\usman[1]{{\color{purple}{{UA: #1}}}}
\makeatother
%%%END OF FIGURE SETUP%%%

%%%TITLE, AUTHORS AND ABSTRACT%%%
\twocolumn[
  \begin{@twocolumnfalse}
{\includegraphics[height=30pt]{head_foot/journal_name}\hfill\raisebox{0pt}[0pt][0pt]{\includegraphics[height=55pt]{head_foot/RSC_LOGO_CMYK}}\\[1ex]
\includegraphics[width=18.5cm]{head_foot/header_bar}}\par
\vspace{1em}
\sffamily
\begin{tabular}{m{4.5cm} p{13.5cm} }

\includegraphics{head_foot/DOI} & \noindent\LARGE{\bf Effects of hydrogen bonding on the $\pi$ depletion and $\pi-\pi$ stacking interactions$^\dag$
} \\%Article title goes here instead of the text "This is the title"
\vspace{0.3cm} & \vspace{0.3cm} \\

 & \noindent\large{Usman Ahmed,\textit{$^{a}$} Dage Sundholm,$^{\ast}$\textit{$^{a}$} and Mikael P.\ Johansson\textit{$^{a,b}$}} \\%Author names go here instead of "Full name", etc.



\includegraphics{head_foot/dates} & \noindent\normalsize{Non-covalent
interactions such as hydrogen bonding and $\pi$-$\pi$ stacking are very
important type of interactions governing molecular self-assembly.  The
$\pi$-$\pi$ stacking ability of aromatic rings depends on the electron density
of the $\pi$ orbitals, which is affected by the electron-withdrawing or
electron-donating properties of the substituents. We have here studied the
effect of hydrogen bonding on the strength of the $\pi$-$\pi$ stacking
interactions by calculating the binding energies at the
M{\o}ller-Plesset (MP2) perturbation theory level using large
basis sets.  The stacking interactions in the presence of hydrogen bonding are
found to be stronger than in the absence of them suggesting that the hydrogen
bonding leads to $\pi$ depletion, which affects the aromatic character of the
aromatic rings and increases the strength of the $\pi$-$\pi$ stacking
interactions.  We have also studied how hydrogen bonding affects the stacking
interaction by calculating local orbital locator integrated pi over plane
(LOLIPOP) indices.
Comparing LOLIPOP indices with the stacking-interaction energies calculated at
MP2 level shows that there is no clear correlation between the
stacking-interaction energies and LOLIPOP indices.}

\end{tabular}

 \end{@twocolumnfalse} \vspace{0.6cm}

  ]
%%%END OF TITLE, AUTHORS AND ABSTRACT%%%

%%%FONT SETUP - please do not change any commands within this section
\renewcommand*\rmdefault{bch}\normalfont\upshape
\rmfamily
\section*{}
\vspace{-1cm}


%%%FOOTNOTES%%%

\footnotetext{\textit{$^{a}$~Department of Chemistry, Faculty of Science, A.I. Virtasen aukio 1, P.O. Box 55, \\ \phantom{.}  FI-00014 University of Helsinki, Finland}}
\footnotetext{\textit{$^{b}$~Center for Scientific Computing, Finland }}

%Please use \dag to cite the ESI in the main text of the article.
%If you article does not have ESI please remove the the \dag symbol from the title and the footnotetext below.

\footnotetext{\dag~Electronic Supplementary Information (ESI) available: The
Cartesian coordinates of the studied molecules and pictures of the molecular
structures of the studied molecules. See DOI: 00.0000/00000000.}

%additional addresses can be cited as above using the lower-case letters, c, d, e... If all authors are from the same address, no letter is required


%%%END OF FOOTNOTES%%%

%%%MAIN TEXT%%%%
\section{Introduction}


The $\pi$-$\pi$ stacking interactions are one of the most commonly investigated
types of non-covalent interactions and they are widely considered to be the
most important type of weak interactions. Stacking interactions between
aromatic rings containing $\pi$ orbitals have been extensively studied during
the past decades because they play a crucial role in a large number of
applications.  The $\pi$-$\pi$ stacking interactions occur in molecular
materials as well as in chemical and biological
systems\cite{https://doi.org/10.1002/anie.200390319} where they stabilize
molecular complexes and
nanostructures\cite{https://doi.org/10.1002/(SICI)1099-1395(199705)10:5<254::AID-POC875>3.0.CO;2-3,doi:10.1021/ar950199y}.
The $\pi$-$\pi$ interactions are responsible for the overall structural
stability\cite{https://doi.org/10.1002/(SICI)1097-0134(20000215)38:3<288::AID-PROT5>3.0.CO;2-7,Burley23,doi:10.1021/ja00170a016,CHAKRABARTI19959},
thermal stability\cite{10.1093/protein/13.11.753} and
folding\cite{10.1093/protein/15.2.91} of proteins. 

The $\pi$-$\pi$ stacking interactions have also been studied for use in
optoelectronic applications of organic crystals, where they were found to be
responsible for
electrical\cite{C2CC37655E,C9SC03348C,doi:10.1021/ja057198d,doi:10.1021/ja066426g,B717752F,B925259B,doi:10.1021/jp208182s,Chen2016}
and thermal conductivity\cite{SU20181} in different devices. Materials with
$\pi$-$\pi$ interactions have also been used for sensing
explosives\cite{C5CC07513K,C4RA12835D} and in sensing dyes\cite{C2OB26117K}.
The $\pi$-$\pi$ interactions guide the favourite aggregation
modes\cite{C9EN01365B}  and are involved in the self-assembly of crystal
structures\cite{VENKATESAN2019306,SHUKLA2017426}. They play a crucial role in
the swelling of lignite, which is a type of low-rank coal\cite{HOU2021120920}.
The stacking-interaction energies improve the healability, self-healing
properties, and the toughness of
materials\cite{doi:10.1021/cm102963k,https://doi.org/10.1002/marc.201700018,doi:10.1021/acssuschemeng.0c04136,doi:10.1021/acs.chemmater.9b01239}.
They also promote the delivery of chemical and biological
drugs\cite{ZHUANG2019311,doi:10.1021/jp806751k,LIANG20151,doi:10.1021/acsami.7b08534,doi:10.1021/acsnano.8b01186,https://doi.org/10.1002/adfm.201002497,SONG2018406,BAO201763,MIAO201528,LIU2016401,C7OB01975K,C8CC04363A,C4CC10226F,doi:10.1021/acsami.8b18425,doi:10.1021/nn501816z,doi:10.1021/acsnano.5b00929,doi:10.1021/ja409686x,thno20028,https://doi.org/10.1002/anie.200902612,https://doi.org/10.1002/adfm.200901245,Ge16968,doi:10.3109/1061186X.2011.639023,10.1093/jnci/djw186}.

Hydrogen bonding is another type of non-covalent interaction that plays an
important role in biological systems as well as in supramolecular
chemistry\cite{B103906G}. Hydrogen bonding has been explored in the
construction of crystalline and polymeric materials exhibiting novel and useful
properties\cite{doi:10.1021/acs.macromol.5b01990,C4PY01715C,C5SC00988J,https://doi.org/10.1002/anie.201707097}.

Hydrogen bonds also affect the strength of $\pi$-$\pi$ interactions because the
cyclic $\pi$-electron delocalization of aromatic rings may limit an efficient
stacking, whereas a stronger interaction can be obtained by depleting the
$\pi$-electron density and thereby reducing the degree of aromaticity
\cite{Bloom:11}.  

In this work, we investigate how hydrogen bonding interactions affect the
$\pi$-stacking ability of aromatic rings.  We have studied 17 aromatic
molecules whose dimers are linked with quadruple hydrogen bonding.  Seven
dimers have the DDAA-AADD motif, where D denotes the donor and A is the
acceptor of the hydrogen bond.  Ten of the dimers have the DADA-ADAD motif.
Examples of the DDAA-AADD and DADA-ADAD motifs are shown in \cref{fig:figure1}.
More pictures of the molecular structures are
reported in the Electronic Supplemental Information (ESI)\dag.

%The previously accepted notion was that the secondary A$\cdots$D interactions
%are attractive, whereas the secondary A$\cdots$A and D$\cdots$D interactions
%are repulsive. The DDAA-AADD hydrogen-bonding structures are therefore expected
%to bind stronger than the DADA-ADAD ones\cite{C3RA43814G}. 
%However, in our
%recent study\cite{Usman:paper-1}, we showed that the DDAA-AADD hydrogen bonding
%motifs have five attractive (D$\cdots$A and A$\cdots$A) and one repulsive
%(D$\cdots$D) secondary interaction, while the DADA-ADAD motif have three
%attractive (A$\cdots$A) and three repulsive (D$\cdots$D) secondary
%interactions. The DDAA-AADD hydrogen bonding motif still binds stronger than
%the DADA-ADAD motif\cite{Usman:paper-1}.

We have here calculated the strength of $\pi$-stacking interactions and
hydrogen-bonding strengths at the second-order
M{\o}ller-Plesset (MP2) perturbation theory level using large basis sets.
The obtained hydrogen-bonding strengths are compared to benchmark values that
we have recently obtained in large-scale coupled-cluster
calculations\cite{Usman:paper-1}. We have also employed the LOLIPOP (local
orbital locator integrated pi over plane) approach to estimate the extent of
the $\pi$ depletion and the $\pi$-stacking ability of the studied molecules.  

We describe the employed computation methods in Section \ref{sec:compmeth}.
The results are discussed in the Section \ref{sec:results}, which is followed
by brief conclusions in Section \ref{sec:conclusions}. 


\begin{figure}[H]
\centering
\subfloat[a][]{\includegraphics[width=0.47\columnwidth]{figure1a.pdf} \label{fig:DDAA04}}  \hspace{2mm} 
\subfloat[b][]{\includegraphics[width=0.47\columnwidth]{figure1b.pdf} \label{fig:DADA01}}
\caption{(a) DDAA04 is a quadruple hydrogen-bonded dimer with the more
favourable DDAA-AADD hydrogen-bonding pattern and (b) DADA01 has the
less favourable DADA-ADAD hydrogen-bonding pattern. The figure was made with Jmol\cite{jmol}.} 
\label{fig:figure1}
\end{figure}



\section{Computational methods}
\label{sec:compmeth}

The molecular structures of the studied molecules, their dimers and the benzene
probe were optimized with Turbomole\cite{https://doi.org/10.1002/wcms.1162} at
the density functional theory (DFT) level using the TPSSh
functional\cite{doi:10.1063/1.1626543,doi:10.1063/1.1795692}, the aug-cc-pVTZ
basis sets\cite{B515623H,B508541A} and the D3(BJ) dispersion
correction\cite{doi:10.1063/1.3382344,https://doi.org/10.1002/jcc.21759}.  The
molecular structures are shown in the ESI\dag. The Cartesian coordinates of the
optimized molecular structures of the studied molecules, their dimers, and the
molecules with benzene probe are also given in the ESI\dag.  The stacking
interaction energies have been calculated for 17 of the studied systems at the
MP2 level using Davidson's contraction of the aug-cc-pVQZ basis
sets\cite{Dunning:89,doi:10.1063/1.456153,DAVIDSON1996514}. For the two largest
molecules and their dimers, we employed the aug-cc-pVTZ basis sets in the
MP2 calculations because the dimers are large.  The hydrogen-bonding
interactions in \cref{tab:table1} were calculated at the MP2 level using
the aug-cc-pVQZ basis sets\cite{doi:10.1063/1.1627293}.

The hydrogen bonding leads to $\pi$ depletion that enhances the $\pi$-stacking
ability of aromatic rings\cite{C2CC33886F}. Population analysis was used for
quantifying the $\pi$ depletion of the aromatic rings of the dimers. The
optimized structures were employed in the population analysis using
the TPSSh functional\cite{doi:10.1063/1.1626543,doi:10.1063/1.1795692} and
the aug-cc-pVQZ\cite{doi:10.1063/1.456153,DAVIDSON1996514} basis sets.

The stacking-interaction energies were estimated by single-point calculations
of the interaction energy between the studied systems and a benzene probe
placed at a distance of about 3.8 {\AA} above aromatic rings as shown in
\cref{fig:figure2}. The effect on the stacking-interaction energies due to
hydrogen bonding were estimated by comparing stacking energies calculated for
the monomers and dimers, respectively.  

\begin{figure}[H] \centering
\subfloat[a][DDAA04-R]{\includegraphics[width=0.47\columnwidth]{figure2a.pdf}
\label{fig:DDAA04-2}}  \hspace{2mm}
\subfloat[b][DADA01]{\includegraphics[width=0.47\columnwidth]{figure2b.pdf}
\label{fig:DADA01-1}}
\caption{The benzene probe above an aromatic ring is shown for (a) DDAA04-R, which is a monomer with the DDAA motif and for (b) DADA01, which is a monomer 
with the DADA motif. The figure was made with Jmol\cite{jmol}.}
\label{fig:figure2} 
\end{figure}

The effect of the hydrogen bonding on the stacking energy was also estimated by
calculating the $\pi$ depletion using the local orbital locator (LOL) approach.
LOL is a function of the kinetic energy density that can be used for depicting
$\pi$ bonds and for describing the nature and location of electron
pairs\cite{C2CC33886F}. LOLIPOP indices, which quantify the $\pi$ depletion,
were obtained by integrating the LOL function over a spatial domain consisting
of a cylinder with a radius of 1.94 {\AA}.  The cylinder is placed in the
middle of the benzene ring covering the main part of the $\pi$-electron
density\cite{C2CC33886F}. The LOLIPOP analyses were performed using the
Multiwfn software package\cite{https://doi.org/10.1002/jcc.22885} with the
molecular orbital file in Molden\cite{molden} format as input. The Multiwfn
input file was created with
ORCA\cite{https://doi.org/10.1002/wcms.81,https://doi.org/10.1002/wcms.1327} at
the TPSSh level of theory using the def2-SVP basis
sets\cite{doi:10.1063/1.1626543,doi:10.1063/1.1795692,Schafer:92}.


\section{Results and Discussion}
\label{sec:results}

\begin{table}[h] 
\small 
\caption{Comparison of the hydrogen-bonding energies (in kJ/mol) of the studied
molecules calculated at the MP2/aug-cc-pVQZ and CCSD(T)/aug-cc-pVQZ 
levels.} 
\label{tab:table1}
\begin{threeparttable}
\begin{tabularx}{0.48\textwidth}{LCC}
\hline
\multicolumn{1}{l}{Dimer} & {MP2} & {CCSD(T)}$^a$ \\
\hline 
DDAA01$^b$ & -200.6 & -210.8    \\
DDAA02 & -224.5 & -222.7    \\
DDAA03 & -157.9 & -141.0    \\
DDAA04 & -135.9 & -129.0    \\
DDAA05 & -154.4 & -153.0    \\
DDAA06 & -296.6 & -293.3    \\
DDAA07 & -231.2 & -229.0    \\
DDAA08 & -207.9 & -211.0    \\
DADA01 & -135.0 & -132.8    \\
DADA02 & -123.5 & -121.3    \\
DADA03 & -124.9 & -122.8    \\
DADA04 & -118.9 & -115.5    \\
DADA05 & -122.2 & -118.0    \\
DADA06 & -137.3 & -129.6    \\
DADA07 & -138.6 & -131.4    \\
DADA08 & -133.4 & -132.3    \\
DADA09$^b$ & -187.0 & -176.7    \\
DADA10 & -127.0 & -124.2    \\

\hline
\end{tabularx}
\begin{tablenotes}
\item[a] The reference CCSD(T) data are taken from Ref.\
\citenum{Usman:paper-1}.
\item[b] The CCSD(T) and MP2 binding energies for DDAA01 and DADA09 were calculated
using the aug-cc-pVTZ basis sets.
\end{tablenotes}
\end{threeparttable}
\end{table}

\subsection{Hydrogen bonding energies of the dimers}
\label{sec:H-bond-energy}

The binding energies of the quadruple hydrogen bonded dimers calculated at the
MP2 level are compared in \cref{tab:table1} to reference values calculated
at the coupled-cluster singles and doubles level with a perturbational
treatment of the triples (CCSD(T)) level using large basis
sets\cite{Usman:paper-1}. The naming scheme follows the notation of the
hydrogen bonding motifs.  An example of a molecule with the DDAA-AADD hydrogen
bonding motif is shown in \cref{fig:DDAA04}. We call this class of molecules
DDAA$n$, where $n$ is the running numbering of the molecules.  A molecule with
the DADA-ADAD motif representing the DADA$n$ molecules is shown in
\cref{fig:DADA01}. 


The hydrogen bonding energies calculated at the MP2 level agree well with
the CCSD(T) reference energies.  A linear regression fit yields an angular
coefficient of $1.055\pm0.088$ and an offset of $-27.3\pm11.6$ kJ/mol showing
that the MP2 and CCSD(T) energies correlate well. However, the hydrogen
bonding energies calculated at the MP2 level are systematically smaller
than the CCSD(T) reference values. The binding energies of the DDAA01 and
DADA09 dimers  in \cref{tab:table1} were calculated with smaller basis sets
because the reference CCSD(T) energies could not be computed with the larger
basis set.  Since the hydrogen bonding energies calculated at the MP2 level
agree well with the CCSD(T) data, the stacking energies calculated at the
MP2 level are also expected to be accurate. 


\subsection{Stacking interaction energy}
\label{sec:energy}

The $\pi$ depletion of aromatic rings is related to the $\pi$-stacking ability
of the molecules\cite{C2CC33886F,Bloom:11}. Hydrogen bonding leads to charge
transfer from the donor (D) to the acceptor (A) of the hydrogen bond. It also
leads to a localization of the electronic charge to the vicinity of the
hydrogen bond implying that hydrogen bonding involving aromatic rings leads to
charge transfer from the ring towards the hydrogen bond and thereby
strengthening the stacking interaction.  We study here the effect of hydrogen
bonding on the $\pi$ depletion and consequently on the molecular $\pi$-stacking
energies.  The stacking energies for the DDAA-AADD and DADA-ADAD hydrogen
bonding motifs calculated at the MP2 level are summarized in
\cref{tab:table2,tab:table3}, respectively. Changes in the stacking energies
due to the hydrogen bonding were obtained by calculating the $\pi$-stacking
energies for the monomers and the dimers.  The stacking energies were obtained
in single-point calculations with a benzene probe placed at a distance of 3.8
{\AA} above the aromatic rings of the molecules.  The obtained stacking
energies of the DDAA$n$ molecules and their dimers are given in
\cref{tab:table2}.  Some of the molecules have more than one aromatic ring as
seen \textit{e.g.}, in \cref{fig:DDAA04}.  We introduce the label L (left) and
R (right) to denote the studied aromatic ring of the DDAA and DADA molecules
with two aromatic rings.  The molecules are oriented with the first D to the
left. 

The $\pi$-stacking ability of the aromatic rings of the dimers is stronger
because the dimers are complexed through hydrogen bonding to their identical
copy.  This pattern can be clearly seen in \cref{tab:table2} where the
$\pi$-stacking interaction energies are larger for the hydrogen bonded dimers
than for the monomers.

\begin{table}[h] 
\small 
\caption{Comparison of the stacking interaction energies (in kJ/mol) and
LOLIPOP values of the DDAA-AADD molecules with and without the presence of
hydrogen bonding. Molecule DDAA05 is omitted because it has no aromatic rings.}
\label{tab:table2}
	\begin{threeparttable}
\begin{tabularx}{0.48\textwidth}{LLCCCC}
\hline
  &       & \multicolumn{2}{c}{{Dimer} } 
         & \multicolumn{2}{c}{{Monomer}} \\   
\multicolumn{2}{l}{Molecule} & {Energy} 
         & {LOLIPOP} 
         & {Energy} 
         & {LOLIPOP}\\
\hline 
\multicolumn{2}{l}{DDAA01-R\tnote{a}} & -37.79  & 0.85  & -34.84 & 1.96 \\
\multicolumn{2}{l}{DDAA01-L\tnote{a}} & -27.15  & 1.80  & -26.13 & 2.14 \\
\multicolumn{2}{l}{DDAA02} &	        -27.53  & 3.63  & -23.32 & 4.03 \\
\multicolumn{2}{l}{DDAA03-R} &	        -28.95  & 2.32  & -24.41 & 2.69 \\
\multicolumn{2}{l}{DDAA03-L} &	        -33.06  & 1.83  & -27.17 & 3.41 \\
\multicolumn{2}{l}{DDAA04-L} &	        -34.36  & 4.62  & -28.43 & 4.69 \\
\multicolumn{2}{l}{DDAA04-R} &	        -31.73  & 3.03  & -27.12 & 3.14 \\
\multicolumn{2}{l}{DDAA06-L} &	        -26.91  & 4.75  & -25.18 & 4.98 \\
\multicolumn{2}{l}{DDAA06-R} &	        -29.42  & 4.75  & -25.84 & 4.94 \\
\multicolumn{2}{l}{DDAA07} &	        -30.35  & 3.68  & -25.44 & 4.09 \\
\multicolumn{2}{l}{DDAA08-R} &	        -27.96  & 6.15  & -26.10 & 6.25 \\
\multicolumn{2}{l}{DDAA08-L} &	        -35.06  & 3.17  & -29.64 & 3.59 \\
\hline
\end{tabularx}
\begin{tablenotes}
\item[a] The binding energies for DDAA01 were  
calculated using the aug-cc-pVTZ basis sets. 
\end{tablenotes}
	\end{threeparttable}
\end{table}

The left and right aromatic rings of DDAA01 exhibit different $\pi$-stacking
strengths because the rings have different chemical environment. The stacking
energy of the right ring of the DDAA01 dimer is -34.32 kJ/mol, whereas the
stacking energy of the same ring of the DDAA01 monomer is -31.45 kJ/mol.  The
difference of 2.87 kJ/mol in stacking energy is due to the $\pi$ depletion
caused by hydrogen bonds. 

\begin{table}[h]
\small
\caption{Comparison of the stacking interaction energies (in kJ/mol) and
LOLIPOP values of the DADA-ADAD molecules with and without the presence of
hydrogen bonding.}
\label{tab:table3}
\begin{threeparttable}
\begin{tabularx}{0.48\textwidth}{LLCCCC}
\hline
  &       & \multicolumn{2}{c}{{Dimer} } 
         & \multicolumn{2}{c}{{Monomer}} \\   
\multicolumn{2}{l}{Molecule} & {Energy} 
         & {LOLIPOP} 
         & {Energy} 
         & {LOLIPOP}\\















\hline
\hline 
\multicolumn{2}{l}{DADA01}            & -35.06 & 4.24 & -28.19 & 4.33 \\
\multicolumn{2}{l}{DADA02}            & -22.60 & 2.80 & -21.46 & 2.78 \\
\multicolumn{2}{l}{DADA03}            & -26.93 & 4.09 & -24.33 & 4.19 \\
\multicolumn{2}{l}{DADA04}            & -30.56 & 4.47 & -26.74 & 4.56 \\
\multicolumn{2}{l}{DADA05-R}          & -27.19 & 3.06 & -26.41 & 3.10 \\
\multicolumn{2}{l}{DADA05-L}          & -32.59 & 3.30 & -29.34 & 3.31 \\
\multicolumn{2}{l}{DADA06-R}          & -24.57 & 3.67 & -22.26 & 4.71 \\
\multicolumn{2}{l}{DADA06-L}          & -22.80 & 4.55 & -22.18 & 5.67 \\
\multicolumn{2}{l}{DADA07-R}          & -38.01 & 3.01 & -36.55 & 3.92 \\
\multicolumn{2}{l}{DADA07-L}          & -35.10 & 3.84 & -33.46 & 5.06 \\
\multicolumn{2}{l}{DADA08}            & -30.86 & 4.35 & -27.86 & 4.41 \\
\multicolumn{2}{l}{DADA09-L\tnote{a}} & -30.57 & 1.22 & -26.97 & 2.73 \\
\multicolumn{2}{l}{DADA09-R\tnote{a}} & -24.02 & 1.45 & -20.24 & 2.05 \\
\multicolumn{2}{l}{DADA10-L}          & -30.33 & 4.82 & -26.19 & 5.02 \\
\multicolumn{2}{l}{DADA10-R}          & -30.11 & 4.60 & -26.03 & 4.70 \\
\hline
\end{tabularx}
\begin{tablenotes}
\item[a] The binding energies for DADA09 were 
calculated using the aug-cc-pVTZ basis sets.
\end{tablenotes}
\end{threeparttable}
\end{table}

The stacking interaction energies of the monomers and dimers of the molecules
with the DADA-ADAD hydrogen bonding motif are compared in \cref{tab:table3}.
The same trend is obtained for the  molecules with the DADA-ADAD hydrogen
bonding motif as for the DDAA-AADD ones.  The difference in the $\pi$-stacking
energies is somewhat larger for the DDAA-AADD molecules than for the DADA-ADAD
ones because the DDAA-AADD dimers have stronger hydrogen bonds than the
DADA-ADAD ones.  Stronger hydrogen bonds leads to a larger $\pi$ depletion and
a stronger stacking interaction.

\subsection{Population Analysis}

A population analysis of the studied molecules was performed to corroborate the
$\pi$ depletion due to the hydrogen bonds of the dimers.  The hydrogen bonds
attract electrons from the nearest atoms of the aromatic rings. The electron
attraction is stronger for the acceptor (A) part of the hydrogen bond than for
the donor (D). The aromatic ring is for most of the studied molecules the
acceptor of the hydrogen bond. The atoms of the aromatic rings donate electrons
to the hydrogen bond making them less negatively charged in the dimer as
compared to the monomer.

The electronic charges of the rings of the studied molecules are reported in
\cref{tab:table4,tab:table5}. The electronic charges are calculated for all
atoms of the aromatic rings except for the one that is part of the hydrogen
bond.  Some of the studied molecules have fused aromatic rings.  The electronic
charge is then calculated separately for each ring implying that there is a
double counting of the charges of the two common carbon atoms of the rings. 

\begin{table}[h]
\small
\caption{Comparison of the charges (in $e$) of the aromatic rings 
with and without the presence of hydrogen bonding for the DDAA-AADD
motif. The acceptor atom is excluded in the population analysis.}
\label{tab:table4}
        \begin{threeparttable}
\begin{tabularx}{0.48\textwidth}{LLCCCC}
\hline
	
	\multicolumn{2}{l}{Molecule}  &  {Dimer} & {Monomer}
         & {Difference} \\
\hline
\multicolumn{2}{l}{DDAA01-R\tnote{a}} & -0.17 & -0.19  & 0.02  \\
\multicolumn{2}{l}{DDAA01-L\tnote{a}} & -0.64 & -0.64  & 0.00  \\
\multicolumn{2}{l}{DDAA02} & 0.35 & 0.26  & 0.09  \\
\multicolumn{2}{l}{DDAA03-R} & -0.12 & -0.15  & 0.03  \\
\multicolumn{2}{l}{DDAA03-L} & 0.25 & 0.18  & 0.07  \\
\multicolumn{2}{l}{DDAA04-R} & -0.04 & -0.06  & 0.02  \\
\multicolumn{2}{l}{DDAA04-L} & 0.38 & 0.31  & 0.07  \\
\multicolumn{2}{l}{DDAA06-R} & 0.35 & 0.32  & 0.03  \\
\multicolumn{2}{l}{DDAA06-L} & 0.23 & 0.16  & 0.07  \\
\multicolumn{2}{l}{DDAA07} & 0.53 & 0.44  & 0.09 \\
\multicolumn{2}{l}{DDAA08-R} & -0.56 & -0.59  & 0.03 \\
\multicolumn{2}{l}{DDAA08-L} & 0.84 & 0.77  & 0.07 \\
\hline
\end{tabularx}
\begin{tablenotes}
\item[a] The charges for DDAA01 were
calculated using the def2-TZVPP basis sets.
\end{tablenotes}
        \end{threeparttable}
\end{table}

\Cref{tab:table4} summarizes the charge of the rings for the DDAA-AADD
molecules.  For example, the DDAA01-R ring has a charge of \mbox{-0.17} $e$ and
-0.19 $e$ in the dimer and the monomer, respectively, whereas the DDAA01-L ring
exhibits no change in the charge when forming the dimer. The DDAA01-R and the
DDAA01-L rings do not participate in the hydrogen bonding, whereas on both sides
of the rings there are acceptor and donor atoms leading to charge cancellation.
The acceptor draws more electronic charge from the neighbouring atoms than the
donor and therefore DDAA01-L exhibits no change in the charge when the monomers
form the dimer.  The hydrogen bonding of the dimers results in $\pi$ depletion
making the atoms of the ring near the donor of the hydrogen bond less
negatively charged.  The same pattern can be seen for all rings of the studied
molecules except for DDAA01-L as discussed above.


\Cref{tab:table5} summarizes the charge of the aromatic rings of the DADA-ADAD
molecules.  The same trend is obtained for them as for DDAA-AADD.  Aromatic
rings are less negatively charged in the dimer than in the monomer.  The
DADA05-L, DADA06-R, DADA06-L and DADA07-L rings do not participate in the
hydrogen bonding. When they form the dimer, the charges of the ring change by
only 0.00, 0.00, -0.02 and -0.01 $e$, respectively. DADA05-L is on the opposite
side of the molecule with respect to the hydrogen bonds implying that its
charge does not change upon dimerization. The aromatic rings of DADA06 are
distant from the hydrogen bonds. The left aromatic ring of DADA06 is further
away from the hydrogen bond and is next to the donor. 
The right aromatic ring of DADA07 participates in the hydrogen bonding by
donating a hydrogen, however, the ring is next to the carbonyl and hydroxyl
functional groups. The presence of these functional groups leads to a resonance
structure that directs more charge to the ring in the presence of hydrogen
bonding. 





\begin{table}[h]
\small
\caption{Comparison of the charges (in $e$) of the aromatic rings 
with and without the presence of hydrogen bonding for the DADA-ADAD
motif. The acceptor atom is excluded in the population analysis.}
\label{tab:table5}
\begin{threeparttable}
\begin{tabularx}{0.48\textwidth}{LLCCCC}
\hline
\multicolumn{2}{l}{Molecule} & {Dimer}
	 & {Monomer} & {Difference} \\
\hline
\hline
	\multicolumn{2}{l}{DADA01} & 0.21 & 0.18 & 0.03 \\
	\multicolumn{2}{l}{DADA02} & 0.32 & 0.25 & 0.07 \\
	\multicolumn{2}{l}{DADA03} & 0.12 & 0.11  & 0.01\\
	\multicolumn{2}{l}{DADA04} & 0.22 & 0.19  & 0.03\\
	\multicolumn{2}{l}{DADA05-R} & 0.67 & 0.66  & 0.01\\
	\multicolumn{2}{l}{DADA05-L} & -0.50 & -0.50  & 0.00\\
	\multicolumn{2}{l}{DADA06-R} & -1.08 & -1.08  & 0.00\\
	\multicolumn{2}{l}{DADA06-L} & -1.10 & -1.08  & -0.02\\
	\multicolumn{2}{l}{DADA07-R} & 0.21 & 0.22  & -0.01\\
	\multicolumn{2}{l}{DADA07-L} & -0.21 & -0.22  & 0.01\\
	\multicolumn{2}{l}{DADA08} & 0.18 & 0.16 &   0.02\\
	\multicolumn{2}{l}{DADA09-L\tnote{a}} & -0.39 & -0.41 & 0.02\\
	\multicolumn{2}{l}{DADA09-R\tnote{a}} & -0.39 & -0.41 & 0.02\\
	\multicolumn{2}{l}{DADA10-L} & 0.19 & 0.15   & 0.04\\
	\multicolumn{2}{l}{DADA10-R} & 0.37 & 0.35   & 0.02\\
\hline
\end{tabularx}
\begin{tablenotes}
\item[a] The charges for DADA09 were
calculated using the def2-TZVPP basis sets.
\end{tablenotes}
\end{threeparttable}
\end{table}


\subsection{LOLIPOP}
\label{sec:lolipop}

The extent of the $\pi$ depletion is also estimated using the local orbital
locator (LOL) approach. LOL is a function of kinetic energy density.
LOL$_{\pi}$ describes $\pi$ depletion and provides information about the nature
and location of electron pairs\cite{C1CP21055F}.  The LOLIPOP index is a 
measure of the extent of ${\pi}$ depletion reflecting the number and size of the
LOL$_{\pi}$ isosurfaces of the aromatic rings and can thus be used for 
quantifying the $\pi$-stacking ability. 
Large LOLIPOP values mean less $\pi$ depletion which ultimately means a lower
$\pi$-stacking ability, whereas smaller LOLIPOP values indicate that there is
more $\pi$ depletion and the $\pi$-stacking ability is expected to be better. 


\begin{figure}[H] \centering
\includegraphics[width=0.97\columnwidth]{figure3.pdf}
\caption{The stacking energy (in kJ/mol) of the aromatic rings of the monomers (M) and dimers (D) with the DDAA and DADA motif as a function of the LOLIPOP index. The figure was made with GnuPlot\cite{gnuplot}.} 
\label{fig:figure3} 
\end{figure}

The LOLIPOP indices were calculated for the same aromatic rings of the monomer
and the dimer.  The first aim of the calculations was to investigate how the
LOLIPOP values change when the molecules form dimers via hydrogen bonding.
LOLIPOP indices for the molecules with the DDAA-AADD and DADA-ADAD hydrogen
bonding motifs are summarized in \cref{tab:table2,tab:table3}, respectively.
It is evident from the \cref{tab:table2} that the LOLIPOP indices decrease in
the presence of the hydrogen bonds suggesting a depletion of $\pi$ electrons
and stronger $\pi$-stacking interaction of the dimers. All rings of the
molecules follow the same trend. The LOLIPOP index for the right ring of the
DDAA01 monomer is 1.96 and it decreases to 0.85 when it forms the dimer. The
difference of 1.11 authenticates $\pi$ depletion.  Similarly, the left ring of
the DDAA04 monomer has a LOLIPOP index of 3.03 and it is 3.14 for the dimer.
The LOLIPOP indices and the $\pi$-stacking energies predict $\pi$ depletion.

The reliability of the LOLIPOP index for estimating $\pi$-stacking energies was
studied by calculating the stacking energy as a function of the LOLIPOP index,
which is shown in \cref{fig:figure3}. A linear fit to the points in the graph
yields an angular coefficient of $1.19 \pm 0.39$ kJ/mol/LI and an intercept of
$-26.73 \pm 1.52$ kJ/mol, where LI denotes the LOLIPOP value.  The LOLIPOP
calculations yield largely the correct general trend \textit{i.e.}, a larger
LOLIPOP value means a weaker $\pi$-$\pi$ interaction. However, there is not a
very clear correlation between the the binding energy and the LOLIPOP index as
seen in \cref{fig:figure3} and \cref{tab:table2,tab:table3}.

\section{Conclusions}
\label{sec:conclusions}

The effect of the hydrogen bonding on the $\pi$-stacking interaction of aromatic
rings has been studied for quadruple hydrogen-bonded dimers.  The hydrogen
bonds lead to a transfer of $\pi$ electrons from the aromatic rings towards the
hydrogen bonds, which strengthens $\pi$-stacking interactions.  We employed the
MP2 level of theory to calculate the energies of the hydrogen bonds and the
$\pi$-stacking energies.  The obtained strengths of the hydrogen bonds are
compared to the benchmark energies computed at the CCSD(T)
level\cite{Usman:paper-1}. The strengths of the hydrogen bonds calculated at
the MP2 level correlate well with the ones obtained in the  CCSD(T)
calculations. The CCSD(T) binding energies are -27.3 $\pm$ 11.6 kJ/mol stronger
than the MP2 ones. The agreement between the CCSD(T) and MP2 energies
suggest that also $\pi$-stacking energies calculated at the MP2 level are
accurate. The aromatic rings of the dimers exhibit stronger $\pi$-stacking
interactions compared to the monomers as shown in \cref{tab:table2,tab:table3}.
Molecules with the DDAA-AADD and DADA-ADAD hydrogen-bonding motifs follow the
same pattern. However, molecules with the DDAA-AADD hydrogen-bonding motif
have stronger binding energies than the ones with the DADA-ADAD motif.

We also used the local orbital locator integrated pi over plane (LOLIPOP)
approach and population analysis to estimate the depletion of $\pi$ electrons
from aromatic rings. The results obtained with the LOLIPOP approach also
confirm that $\pi$ electrons of the aromatic rings are transferred toward the
hydrogen bonds. The population analysis shows that the atoms of the aromatic
rings exhibit less electron charge in the dimers than in the monomers.

The MP2 level of theory has proven to be an appropriate method for
calculating the $\pi$-stacking energies for the monomers and the dimers.  On
the other hand, the LOLIPOP indices do not correlate well with the stacking
energies. The LOLIPOP approach can to to some extent be used for quantifying
$\pi$-electron depletion of the aromatic rings. Natural population analysis
yields a general notion about the extent of the charge transfer from the
aromatic rings towards the hydrogen bonds, which leads to a stronger
$\pi$-stacking ability of the rings.

\section*{Author Contributions} 

UA has performed the calculations under supervision of MPJ and DS. The results
have been analyzed by UA and DS. All authors have contributed to writing the 
article.
 
\section*{Conflicts of interest}

There are no conflicts to declare.

\section*{Acknowledgements}

The work has been supported by the Academy of Finland through project numbers
314821 and 340583, by the Magnus Ehrnrooth Foundation, Oskar {\"O}flund
Foundation and by the Swedish Cultural Foundation in Finland. We acknowledge
computational resources from CSC - IT Center for Science, Finland. 


%%%END OF MAIN TEXT%%%

%The \balance command can be used to balance the columns on the final page if desired. It should be placed anywhere within the first column of the last page.

\balance

%If notes are included in your references you can change the title from 'References' to 'Notes and references' using the following command:
%\renewcommand\refname{Notes and references}

%%%REFERENCES%%%
\bibliography{rsc} %You need to replace "rsc" on this line with the name of your .bib file
%\bibliographystyle{rsc} %the RSC's .bst file


\providecommand*{\mcitethebibliography}{\thebibliography}
\csname @ifundefined\endcsname{endmcitethebibliography}
{\let\endmcitethebibliography\endthebibliography}{}
\begin{mcitethebibliography}{79}
\providecommand*{\natexlab}[1]{#1}
\providecommand*{\mciteSetBstSublistMode}[1]{}
\providecommand*{\mciteSetBstMaxWidthForm}[2]{}
\providecommand*{\mciteBstWouldAddEndPuncttrue}
  {\def\EndOfBibitem{\unskip.}}
\providecommand*{\mciteBstWouldAddEndPunctfalse}
  {\let\EndOfBibitem\relax}
\providecommand*{\mciteSetBstMidEndSepPunct}[3]{}
\providecommand*{\mciteSetBstSublistLabelBeginEnd}[3]{}
\providecommand*{\EndOfBibitem}{}
\mciteSetBstSublistMode{f}
\mciteSetBstMaxWidthForm{subitem}
{(\emph{\alph{mcitesubitemcount}})}
\mciteSetBstSublistLabelBeginEnd{\mcitemaxwidthsubitemform\space}
{\relax}{\relax}

\bibitem[Meyer \emph{et~al.}(2003)Meyer, Castellano, and
  Diederich]{https://doi.org/10.1002/anie.200390319}
E.~A. Meyer, R.~K. Castellano and F.~Diederich, \emph{Angew. Chem. Int. Ed.},
  2003, \textbf{42}, 1210--1250\relax
\mciteBstWouldAddEndPuncttrue
\mciteSetBstMidEndSepPunct{\mcitedefaultmidpunct}
{\mcitedefaultendpunct}{\mcitedefaultseppunct}\relax
\EndOfBibitem
\bibitem[Claessens and
  Stoddart(1997)]{https://doi.org/10.1002/(SICI)1099-1395(199705)10:5<254::AID-POC875>3.0.CO;2-3}
C.~G. Claessens and J.~F. Stoddart, \emph{J. Phys. Org. Chem.}, 1997,
  \textbf{10}, 254--272\relax
\mciteBstWouldAddEndPuncttrue
\mciteSetBstMidEndSepPunct{\mcitedefaultmidpunct}
{\mcitedefaultendpunct}{\mcitedefaultseppunct}\relax
\EndOfBibitem
\bibitem[Fyfe and Stoddart(1997)]{doi:10.1021/ar950199y}
M.~C.~T. Fyfe and J.~F. Stoddart, \emph{Acc. Chem. Res.}, 1997, \textbf{30},
  393--401\relax
\mciteBstWouldAddEndPuncttrue
\mciteSetBstMidEndSepPunct{\mcitedefaultmidpunct}
{\mcitedefaultendpunct}{\mcitedefaultseppunct}\relax
\EndOfBibitem
\bibitem[Samanta \emph{et~al.}(2000)Samanta, Pal, and
  Chakrabarti]{https://doi.org/10.1002/(SICI)1097-0134(20000215)38:3<288::AID-PROT5>3.0.CO;2-7}
U.~Samanta, D.~Pal and P.~Chakrabarti, \emph{Proteins: Struct. Funct. Bioinf.},
  2000, \textbf{38}, 288--300\relax
\mciteBstWouldAddEndPuncttrue
\mciteSetBstMidEndSepPunct{\mcitedefaultmidpunct}
{\mcitedefaultendpunct}{\mcitedefaultseppunct}\relax
\EndOfBibitem
\bibitem[Burley and Petsko(1985)]{Burley23}
S.~Burley and G.~Petsko, \emph{Science}, 1985, \textbf{229}, 23--28\relax
\mciteBstWouldAddEndPuncttrue
\mciteSetBstMidEndSepPunct{\mcitedefaultmidpunct}
{\mcitedefaultendpunct}{\mcitedefaultseppunct}\relax
\EndOfBibitem
\bibitem[Hunter and Sanders(1990)]{doi:10.1021/ja00170a016}
C.~A. Hunter and J.~K.~M. Sanders, \emph{J. Am. Chem. Soc.}, 1990,
  \textbf{112}, 5525--5534\relax
\mciteBstWouldAddEndPuncttrue
\mciteSetBstMidEndSepPunct{\mcitedefaultmidpunct}
{\mcitedefaultendpunct}{\mcitedefaultseppunct}\relax
\EndOfBibitem
\bibitem[Chakrabarti and Samanta(1995)]{CHAKRABARTI19959}
P.~Chakrabarti and U.~Samanta, \emph{J. Mol. Biol.}, 1995, \textbf{251},
  9--14\relax
\mciteBstWouldAddEndPuncttrue
\mciteSetBstMidEndSepPunct{\mcitedefaultmidpunct}
{\mcitedefaultendpunct}{\mcitedefaultseppunct}\relax
\EndOfBibitem
\bibitem[Kannan and Vishveshwara(2000)]{10.1093/protein/13.11.753}
N.~Kannan and S.~Vishveshwara, \emph{Protein Eng. Des. Sel.}, 2000,
  \textbf{13}, 753--761\relax
\mciteBstWouldAddEndPuncttrue
\mciteSetBstMidEndSepPunct{\mcitedefaultmidpunct}
{\mcitedefaultendpunct}{\mcitedefaultseppunct}\relax
\EndOfBibitem
\bibitem[Bhattacharyya \emph{et~al.}(2002)Bhattacharyya, Samanta, and
  Chakrabarti]{10.1093/protein/15.2.91}
R.~Bhattacharyya, U.~Samanta and P.~Chakrabarti, \emph{Protein Eng. Des. Sel.},
  2002, \textbf{15}, 91--100\relax
\mciteBstWouldAddEndPuncttrue
\mciteSetBstMidEndSepPunct{\mcitedefaultmidpunct}
{\mcitedefaultendpunct}{\mcitedefaultseppunct}\relax
\EndOfBibitem
\bibitem[Yu \emph{et~al.}(2013)Yu, Wang, Li, Li, Yan, Wang, and
  Pei]{C2CC37655E}
W.~Yu, X.-Y. Wang, J.~Li, Z.-T. Li, Y.-K. Yan, W.~Wang and J.~Pei, \emph{Chem.
  Comm.}, 2013, \textbf{49}, 54--56\relax
\mciteBstWouldAddEndPuncttrue
\mciteSetBstMidEndSepPunct{\mcitedefaultmidpunct}
{\mcitedefaultendpunct}{\mcitedefaultseppunct}\relax
\EndOfBibitem
\bibitem[Xie \emph{et~al.}(2019)Xie, Alexandrov, Skorupskii, Proserpio, and
  Dinc\u{a}]{C9SC03348C}
L.~S. Xie, E.~V. Alexandrov, G.~Skorupskii, D.~M. Proserpio and M.~Dinc\u{a},
  \emph{Chem. Sci.}, 2019, \textbf{10}, 8558--8565\relax
\mciteBstWouldAddEndPuncttrue
\mciteSetBstMidEndSepPunct{\mcitedefaultmidpunct}
{\mcitedefaultendpunct}{\mcitedefaultseppunct}\relax
\EndOfBibitem
\bibitem[Huang and Kertesz(2006)]{doi:10.1021/ja057198d}
J.~Huang and M.~Kertesz, \emph{J. Am. Chem. Soc.}, 2006, \textbf{128},
  1418--1419\relax
\mciteBstWouldAddEndPuncttrue
\mciteSetBstMidEndSepPunct{\mcitedefaultmidpunct}
{\mcitedefaultendpunct}{\mcitedefaultseppunct}\relax
\EndOfBibitem
\bibitem[Huang and Kertesz(2007)]{doi:10.1021/ja066426g}
J.~Huang and M.~Kertesz, \emph{J. Am. Chem. Soc.}, 2007, \textbf{129},
  1634--1643\relax
\mciteBstWouldAddEndPuncttrue
\mciteSetBstMidEndSepPunct{\mcitedefaultmidpunct}
{\mcitedefaultendpunct}{\mcitedefaultseppunct}\relax
\EndOfBibitem
\bibitem[Huang \emph{et~al.}(2008)Huang, Kingsbury, and Kertesz]{B717752F}
J.~Huang, S.~Kingsbury and M.~Kertesz, \emph{Phys. Chem. Chem. Phys.}, 2008,
  \textbf{10}, 2625--2635\relax
\mciteBstWouldAddEndPuncttrue
\mciteSetBstMidEndSepPunct{\mcitedefaultmidpunct}
{\mcitedefaultendpunct}{\mcitedefaultseppunct}\relax
\EndOfBibitem
\bibitem[Tian \emph{et~al.}(2010)Tian, Huang, and Kertesz]{B925259B}
Y.-H. Tian, J.~Huang and M.~Kertesz, \emph{Phys. Chem. Chem. Phys.}, 2010,
  \textbf{12}, 5084--5093\relax
\mciteBstWouldAddEndPuncttrue
\mciteSetBstMidEndSepPunct{\mcitedefaultmidpunct}
{\mcitedefaultendpunct}{\mcitedefaultseppunct}\relax
\EndOfBibitem
\bibitem[Tian and Kertesz(2011)]{doi:10.1021/jp208182s}
Y.-H. Tian and M.~Kertesz, \emph{J. Phys. Chem. A}, 2011, \textbf{115},
  13942--13949\relax
\mciteBstWouldAddEndPuncttrue
\mciteSetBstMidEndSepPunct{\mcitedefaultmidpunct}
{\mcitedefaultendpunct}{\mcitedefaultseppunct}\relax
\EndOfBibitem
\bibitem[Chen \emph{et~al.}(2016)Chen, Gao, and Yang]{Chen2016}
X.~Chen, F.~Gao and W.~Yang, \emph{Sci. Rep.}, 2016, \textbf{6}, 29314\relax
\mciteBstWouldAddEndPuncttrue
\mciteSetBstMidEndSepPunct{\mcitedefaultmidpunct}
{\mcitedefaultendpunct}{\mcitedefaultseppunct}\relax
\EndOfBibitem
\bibitem[Su \emph{et~al.}(2018)Su, Wang, Tian, Huang, Xiao, Guo, He, and
  Tian]{SU20181}
Z.~Su, H.~Wang, K.~Tian, W.~Huang, C.~Xiao, Y.~Guo, J.~He and X.~Tian,
  \emph{Compos. Sci. Technol.}, 2018, \textbf{155}, 1--10\relax
\mciteBstWouldAddEndPuncttrue
\mciteSetBstMidEndSepPunct{\mcitedefaultmidpunct}
{\mcitedefaultendpunct}{\mcitedefaultseppunct}\relax
\EndOfBibitem
\bibitem[Shanmugaraju and Mukherjee(2015)]{C5CC07513K}
S.~Shanmugaraju and P.~S. Mukherjee, \emph{Chem. Comm.}, 2015, \textbf{51},
  16014--16032\relax
\mciteBstWouldAddEndPuncttrue
\mciteSetBstMidEndSepPunct{\mcitedefaultmidpunct}
{\mcitedefaultendpunct}{\mcitedefaultseppunct}\relax
\EndOfBibitem
\bibitem[Liu \emph{et~al.}(2015)Liu, Hao, Shi, Qiu, and Hao]{C4RA12835D}
L.~Liu, J.~Hao, Y.~Shi, J.~Qiu and C.~Hao, \emph{RSC Adv.}, 2015, \textbf{5},
  3045--3053\relax
\mciteBstWouldAddEndPuncttrue
\mciteSetBstMidEndSepPunct{\mcitedefaultmidpunct}
{\mcitedefaultendpunct}{\mcitedefaultseppunct}\relax
\EndOfBibitem
\bibitem[Luisier \emph{et~al.}(2012)Luisier, Ruggi, Steinmann, Favre, Gaeng,
  Corminboeuf, and Severin]{C2OB26117K}
N.~Luisier, A.~Ruggi, S.~N. Steinmann, L.~Favre, N.~Gaeng, C.~Corminboeuf and
  K.~Severin, \emph{Org. Biomol. Chem.}, 2012, \textbf{10}, 7487--7490\relax
\mciteBstWouldAddEndPuncttrue
\mciteSetBstMidEndSepPunct{\mcitedefaultmidpunct}
{\mcitedefaultendpunct}{\mcitedefaultseppunct}\relax
\EndOfBibitem
\bibitem[Tang \emph{et~al.}(2020)Tang, Zhang, Huang, Cui, and Xing]{C9EN01365B}
H.~Tang, S.~Zhang, T.~Huang, F.~Cui and B.~Xing, \emph{Environ. Sci.: Nano},
  2020, \textbf{7}, 984--995\relax
\mciteBstWouldAddEndPuncttrue
\mciteSetBstMidEndSepPunct{\mcitedefaultmidpunct}
{\mcitedefaultendpunct}{\mcitedefaultseppunct}\relax
\EndOfBibitem
\bibitem[Venkatesan \emph{et~al.}(2019)Venkatesan, Cer\'{o}n, Ceballos,
  P\'{e}rez-Guti\'{e}rrez, Thamotharan, and Percino]{VENKATESAN2019306}
P.~Venkatesan, M.~Cer\'{o}n, P.~Ceballos, E.~P\'{e}rez-Guti\'{e}rrez,
  S.~Thamotharan and M.~J. Percino, \emph{J. Mol. Struct.}, 2019,
  \textbf{1196}, 306--322\relax
\mciteBstWouldAddEndPuncttrue
\mciteSetBstMidEndSepPunct{\mcitedefaultmidpunct}
{\mcitedefaultendpunct}{\mcitedefaultseppunct}\relax
\EndOfBibitem
\bibitem[Shukla \emph{et~al.}(2017)Shukla, Mohan, Vishalakshi, and
  Chopra]{SHUKLA2017426}
R.~Shukla, T.~Mohan, B.~Vishalakshi and D.~Chopra, \emph{J. Mol. Struct.},
  2017, \textbf{1134}, 426--434\relax
\mciteBstWouldAddEndPuncttrue
\mciteSetBstMidEndSepPunct{\mcitedefaultmidpunct}
{\mcitedefaultendpunct}{\mcitedefaultseppunct}\relax
\EndOfBibitem
\bibitem[Hou \emph{et~al.}(2021)Hou, Yuchi, Bai, Feng, Guo, Kong, Bai, and
  Li]{HOU2021120920}
R.~Hou, W.~Yuchi, Z.~Bai, Z.~Feng, Z.~Guo, L.~Kong, J.~Bai and W.~Li,
  \emph{Fuel}, 2021, \textbf{299}, 120920\relax
\mciteBstWouldAddEndPuncttrue
\mciteSetBstMidEndSepPunct{\mcitedefaultmidpunct}
{\mcitedefaultendpunct}{\mcitedefaultseppunct}\relax
\EndOfBibitem
\bibitem[Burattini \emph{et~al.}(2011)Burattini, Greenland, Hayes, Mackay,
  Rowan, and Colquhoun]{doi:10.1021/cm102963k}
S.~Burattini, B.~W. Greenland, W.~Hayes, M.~E. Mackay, S.~J. Rowan and H.~M.
  Colquhoun, \emph{Chem. Mater.}, 2011, \textbf{23}, 6--8\relax
\mciteBstWouldAddEndPuncttrue
\mciteSetBstMidEndSepPunct{\mcitedefaultmidpunct}
{\mcitedefaultendpunct}{\mcitedefaultseppunct}\relax
\EndOfBibitem
\bibitem[Zhang \emph{et~al.}(2017)Zhang, Lv, Deng, and
  Wang]{https://doi.org/10.1002/marc.201700018}
G.~Zhang, L.~Lv, Y.~Deng and C.~Wang, \emph{Macromol. Rapid Commun.}, 2017,
  \textbf{38}, 1700018\relax
\mciteBstWouldAddEndPuncttrue
\mciteSetBstMidEndSepPunct{\mcitedefaultmidpunct}
{\mcitedefaultendpunct}{\mcitedefaultseppunct}\relax
\EndOfBibitem
\bibitem[Nie \emph{et~al.}(2020)Nie, Huang, Fan, Cao, Xu, and
  Chen]{doi:10.1021/acssuschemeng.0c04136}
J.~Nie, J.~Huang, J.~Fan, L.~Cao, C.~Xu and Y.~Chen, \emph{ACS Sustain. Chem.
  Eng.}, 2020, \textbf{8}, 13724--13733\relax
\mciteBstWouldAddEndPuncttrue
\mciteSetBstMidEndSepPunct{\mcitedefaultmidpunct}
{\mcitedefaultendpunct}{\mcitedefaultseppunct}\relax
\EndOfBibitem
\bibitem[Chen \emph{et~al.}(2019)Chen, Peng, Thundat, and
  Zeng]{doi:10.1021/acs.chemmater.9b01239}
J.~Chen, Q.~Peng, T.~Thundat and H.~Zeng, \emph{Chem. Mater.}, 2019,
  \textbf{31}, 4553--4563\relax
\mciteBstWouldAddEndPuncttrue
\mciteSetBstMidEndSepPunct{\mcitedefaultmidpunct}
{\mcitedefaultendpunct}{\mcitedefaultseppunct}\relax
\EndOfBibitem
\bibitem[Zhuang \emph{et~al.}(2019)Zhuang, Wang, Cui, Xing, Lee, Kim, Jiang,
  and Oh]{ZHUANG2019311}
W.-R. Zhuang, Y.~Wang, P.-F. Cui, L.~Xing, J.~Lee, D.~Kim, H.-L. Jiang and
  Y.-K. Oh, \emph{J. Control. Release}, 2019, \textbf{294}, 311--326\relax
\mciteBstWouldAddEndPuncttrue
\mciteSetBstMidEndSepPunct{\mcitedefaultmidpunct}
{\mcitedefaultendpunct}{\mcitedefaultseppunct}\relax
\EndOfBibitem
\bibitem[Yang \emph{et~al.}(2008)Yang, Zhang, Liu, Ma, Huang, and
  Chen]{doi:10.1021/jp806751k}
X.~Yang, X.~Zhang, Z.~Liu, Y.~Ma, Y.~Huang and Y.~Chen, \emph{J. Phys. Chem.
  C}, 2008, \textbf{112}, 17554--17558\relax
\mciteBstWouldAddEndPuncttrue
\mciteSetBstMidEndSepPunct{\mcitedefaultmidpunct}
{\mcitedefaultendpunct}{\mcitedefaultseppunct}\relax
\EndOfBibitem
\bibitem[Liang \emph{et~al.}(2015)Liang, Deng, Zhang, Peng, Gao, Cao, Gu, and
  He]{LIANG20151}
Y.~Liang, X.~Deng, L.~Zhang, X.~Peng, W.~Gao, J.~Cao, Z.~Gu and B.~He,
  \emph{Biomaterials}, 2015, \textbf{71}, 1--10\relax
\mciteBstWouldAddEndPuncttrue
\mciteSetBstMidEndSepPunct{\mcitedefaultmidpunct}
{\mcitedefaultendpunct}{\mcitedefaultseppunct}\relax
\EndOfBibitem
\bibitem[Li \emph{et~al.}(2017)Li, Li, Zhou, Lv, Deng, Xu, and
  Yin]{doi:10.1021/acsami.7b08534}
F.~Li, Y.~Li, Z.~Zhou, S.~Lv, Q.~Deng, X.~Xu and L.~Yin, \emph{ACS Appl. Mater.
  Interfaces}, 2017, \textbf{9}, 23586--23601\relax
\mciteBstWouldAddEndPuncttrue
\mciteSetBstMidEndSepPunct{\mcitedefaultmidpunct}
{\mcitedefaultendpunct}{\mcitedefaultseppunct}\relax
\EndOfBibitem
\bibitem[Zeng \emph{et~al.}(2018)Zeng, Zou, Zhang, Wang, Zeng, Cong, and
  Zhang]{doi:10.1021/acsnano.8b01186}
J.-Y. Zeng, M.-Z. Zou, M.~Zhang, X.-S. Wang, X.~Zeng, H.~Cong and X.-Z. Zhang,
  \emph{ACS Nano}, 2018, \textbf{12}, 4630--4640\relax
\mciteBstWouldAddEndPuncttrue
\mciteSetBstMidEndSepPunct{\mcitedefaultmidpunct}
{\mcitedefaultendpunct}{\mcitedefaultseppunct}\relax
\EndOfBibitem
\bibitem[Fung \emph{et~al.}(2011)Fung, Yang, Sadatmousavi, Sheng, Mamo,
  Nazarian, and Chen]{https://doi.org/10.1002/adfm.201002497}
S.-Y. Fung, H.~Yang, P.~Sadatmousavi, Y.~Sheng, T.~Mamo, R.~Nazarian and
  P.~Chen, \emph{Adv. Funct. Mater.}, 2011, \textbf{21}, 2456--2464\relax
\mciteBstWouldAddEndPuncttrue
\mciteSetBstMidEndSepPunct{\mcitedefaultmidpunct}
{\mcitedefaultendpunct}{\mcitedefaultseppunct}\relax
\EndOfBibitem
\bibitem[Song \emph{et~al.}(2018)Song, Guo, Tao, Zhao, Han, and
  Liu]{SONG2018406}
X.~Song, H.~Guo, J.~Tao, S.~Zhao, X.~Han and H.~Liu, \emph{Chem. Eng. Sci.},
  2018, \textbf{187}, 406--414\relax
\mciteBstWouldAddEndPuncttrue
\mciteSetBstMidEndSepPunct{\mcitedefaultmidpunct}
{\mcitedefaultendpunct}{\mcitedefaultseppunct}\relax
\EndOfBibitem
\bibitem[Bao \emph{et~al.}(2017)Bao, Tan, Liang, Zhang, Wang, and
  Liu]{BAO201763}
R.~Bao, B.~Tan, S.~Liang, N.~Zhang, W.~Wang and W.~Liu, \emph{Biomaterials},
  2017, \textbf{122}, 63--71\relax
\mciteBstWouldAddEndPuncttrue
\mciteSetBstMidEndSepPunct{\mcitedefaultmidpunct}
{\mcitedefaultendpunct}{\mcitedefaultseppunct}\relax
\EndOfBibitem
\bibitem[Miao \emph{et~al.}(2015)Miao, Shim, Kim, Lee, Lee, Kim, Byun, and
  Oh]{MIAO201528}
W.~Miao, G.~Shim, G.~Kim, S.~Lee, H.-J. Lee, Y.~B. Kim, Y.~Byun and Y.-K. Oh,
  \emph{J. Control. Release}, 2015, \textbf{211}, 28--36\relax
\mciteBstWouldAddEndPuncttrue
\mciteSetBstMidEndSepPunct{\mcitedefaultmidpunct}
{\mcitedefaultendpunct}{\mcitedefaultseppunct}\relax
\EndOfBibitem
\bibitem[Liu \emph{et~al.}(2016)Liu, Han, Zhang, Yang, Liu, Wang, and
  Wu]{LIU2016401}
N.~Liu, J.~Han, X.~Zhang, Y.~Yang, Y.~Liu, Y.~Wang and G.~Wu, \emph{Colloids
  Surf. B}, 2016, \textbf{145}, 401--409\relax
\mciteBstWouldAddEndPuncttrue
\mciteSetBstMidEndSepPunct{\mcitedefaultmidpunct}
{\mcitedefaultendpunct}{\mcitedefaultseppunct}\relax
\EndOfBibitem
\bibitem[Sun \emph{et~al.}(2017)Sun, Liang, Hao, Xu, Cheng, Su, Cao, Gao, Pu,
  and He]{C7OB01975K}
C.~Sun, Y.~Liang, N.~Hao, L.~Xu, F.~Cheng, T.~Su, J.~Cao, W.~Gao, Y.~Pu and
  B.~He, \emph{Org. Biomol. Chem.}, 2017, \textbf{15}, 9176--9185\relax
\mciteBstWouldAddEndPuncttrue
\mciteSetBstMidEndSepPunct{\mcitedefaultmidpunct}
{\mcitedefaultendpunct}{\mcitedefaultseppunct}\relax
\EndOfBibitem
\bibitem[Cui \emph{et~al.}(2018)Cui, Zhuang, Hu, Xing, Yu, Qiao, He, Li, Ling,
  and Jiang]{C8CC04363A}
P.-F. Cui, W.-R. Zhuang, X.~Hu, L.~Xing, R.-Y. Yu, J.-B. Qiao, Y.-J. He, F.~Li,
  D.~Ling and H.-L. Jiang, \emph{Chem. Comm.}, 2018, \textbf{54},
  8218--8221\relax
\mciteBstWouldAddEndPuncttrue
\mciteSetBstMidEndSepPunct{\mcitedefaultmidpunct}
{\mcitedefaultendpunct}{\mcitedefaultseppunct}\relax
\EndOfBibitem
\bibitem[Deng \emph{et~al.}(2015)Deng, Liang, Peng, Su, Luo, Cao, Gu, and
  He]{C4CC10226F}
X.~Deng, Y.~Liang, X.~Peng, T.~Su, S.~Luo, J.~Cao, Z.~Gu and B.~He, \emph{Chem.
  Comm.}, 2015, \textbf{51}, 4271--4274\relax
\mciteBstWouldAddEndPuncttrue
\mciteSetBstMidEndSepPunct{\mcitedefaultmidpunct}
{\mcitedefaultendpunct}{\mcitedefaultseppunct}\relax
\EndOfBibitem
\bibitem[Yang \emph{et~al.}(2019)Yang, Zhang, Ren, Liu, Zhang, Wang, Huang,
  Zhang, and Liu]{doi:10.1021/acsami.8b18425}
L.~Yang, C.~Zhang, C.~Ren, J.~Liu, Y.~Zhang, J.~Wang, F.~Huang, L.~Zhang and
  J.~Liu, \emph{ACS Appl. Mater. Interfaces}, 2019, \textbf{11}, 331--339\relax
\mciteBstWouldAddEndPuncttrue
\mciteSetBstMidEndSepPunct{\mcitedefaultmidpunct}
{\mcitedefaultendpunct}{\mcitedefaultseppunct}\relax
\EndOfBibitem
\bibitem[Spillmann \emph{et~al.}(2014)Spillmann, Naciri, Algar, Medintz, and
  Delehanty]{doi:10.1021/nn501816z}
C.~M. Spillmann, J.~Naciri, W.~R. Algar, I.~L. Medintz and J.~B. Delehanty,
  \emph{ACS Nano}, 2014, \textbf{8}, 6986--6997\relax
\mciteBstWouldAddEndPuncttrue
\mciteSetBstMidEndSepPunct{\mcitedefaultmidpunct}
{\mcitedefaultendpunct}{\mcitedefaultseppunct}\relax
\EndOfBibitem
\bibitem[Shi \emph{et~al.}(2015)Shi, van~der Meel, Theek, Oude~Blenke, Pieters,
  Fens, Ehling, Schiffelers, Storm, van Nostrum, Lammers, and
  Hennink]{doi:10.1021/acsnano.5b00929}
Y.~Shi, R.~van~der Meel, B.~Theek, E.~Oude~Blenke, E.~H.~E. Pieters, M.~H.
  A.~M. Fens, J.~Ehling, R.~M. Schiffelers, G.~Storm, C.~F. van Nostrum,
  T.~Lammers and W.~E. Hennink, \emph{ACS Nano}, 2015, \textbf{9},
  3740--3752\relax
\mciteBstWouldAddEndPuncttrue
\mciteSetBstMidEndSepPunct{\mcitedefaultmidpunct}
{\mcitedefaultendpunct}{\mcitedefaultseppunct}\relax
\EndOfBibitem
\bibitem[Hu \emph{et~al.}(2013)Hu, Hu, Tian, Ge, Zhang, Luo, and
  Liu]{doi:10.1021/ja409686x}
X.~Hu, J.~Hu, J.~Tian, Z.~Ge, G.~Zhang, K.~Luo and S.~Liu, \emph{J. Am. Chem.
  Soc.}, 2013, \textbf{135}, 17617--17629\relax
\mciteBstWouldAddEndPuncttrue
\mciteSetBstMidEndSepPunct{\mcitedefaultmidpunct}
{\mcitedefaultendpunct}{\mcitedefaultseppunct}\relax
\EndOfBibitem
\bibitem[Wang \emph{et~al.}(2017)Wang, Chen, Xu, Shi, Tayier, Zhou, Zhang, Wu,
  Ye, Fang, and Han]{thno20028}
H.~Wang, J.~Chen, C.~Xu, L.~Shi, M.~Tayier, J.~Zhou, J.~Zhang, J.~Wu, Z.~Ye,
  T.~Fang and W.~Han, \emph{Theranostics}, 2017, \textbf{7}, 3638--3652\relax
\mciteBstWouldAddEndPuncttrue
\mciteSetBstMidEndSepPunct{\mcitedefaultmidpunct}
{\mcitedefaultendpunct}{\mcitedefaultseppunct}\relax
\EndOfBibitem
\bibitem[Liu \emph{et~al.}(2009)Liu, Fan, Rakhra, Sherlock, Goodwin, Chen,
  Yang, Felsher, and Dai]{https://doi.org/10.1002/anie.200902612}
Z.~Liu, A.~Fan, K.~Rakhra, S.~Sherlock, A.~Goodwin, X.~Chen, Q.~Yang,
  D.~Felsher and H.~Dai, \emph{Angew. Chem. Int. Ed.}, 2009, \textbf{48},
  7668--7672\relax
\mciteBstWouldAddEndPuncttrue
\mciteSetBstMidEndSepPunct{\mcitedefaultmidpunct}
{\mcitedefaultendpunct}{\mcitedefaultseppunct}\relax
\EndOfBibitem
\bibitem[Li \emph{et~al.}(2010)Li, Zhu, You, Zhao, Ruan, Zeng, and
  Ding]{https://doi.org/10.1002/adfm.200901245}
F.~Li, Y.~Zhu, B.~You, D.~Zhao, Q.~Ruan, Y.~Zeng and C.~Ding, \emph{Adv. Funct.
  Mater.}, 2010, \textbf{20}, 669--676\relax
\mciteBstWouldAddEndPuncttrue
\mciteSetBstMidEndSepPunct{\mcitedefaultmidpunct}
{\mcitedefaultendpunct}{\mcitedefaultseppunct}\relax
\EndOfBibitem
\bibitem[Ge \emph{et~al.}(2011)Ge, Du, Zhao, Wang, Liu, Li, Yang, Zhou, Zhao,
  Chai, and Chen]{Ge16968}
C.~Ge, J.~Du, L.~Zhao, L.~Wang, Y.~Liu, D.~Li, Y.~Yang, R.~Zhou, Y.~Zhao,
  Z.~Chai and C.~Chen, \emph{Proc. Natl. Acad. Sci. (USA)}, 2011, \textbf{108},
  16968--16973\relax
\mciteBstWouldAddEndPuncttrue
\mciteSetBstMidEndSepPunct{\mcitedefaultmidpunct}
{\mcitedefaultendpunct}{\mcitedefaultseppunct}\relax
\EndOfBibitem
\bibitem[Lai \emph{et~al.}(2012)Lai, Long, Lei, Deng, He, Sheng, Li, and
  Gu]{doi:10.3109/1061186X.2011.639023}
Y.~Lai, Y.~Long, Y.~Lei, X.~Deng, B.~He, M.~Sheng, M.~Li and Z.~Gu, \emph{J.
  Drug. Target.}, 2012, \textbf{20}, 246--254\relax
\mciteBstWouldAddEndPuncttrue
\mciteSetBstMidEndSepPunct{\mcitedefaultmidpunct}
{\mcitedefaultendpunct}{\mcitedefaultseppunct}\relax
\EndOfBibitem
\bibitem[Kim \emph{et~al.}(2016)Kim, Shon, Kim, and Oh]{10.1093/jnci/djw186}
M.-G. Kim, Y.~Shon, J.~Kim and Y.-K. Oh, \emph{J. Natl. Cancer. Inst.}, 2016,
  \textbf{109}, djw186(1--10)\relax
\mciteBstWouldAddEndPuncttrue
\mciteSetBstMidEndSepPunct{\mcitedefaultmidpunct}
{\mcitedefaultendpunct}{\mcitedefaultseppunct}\relax
\EndOfBibitem
\bibitem[Cooke and Rotello(2002)]{B103906G}
G.~Cooke and V.~M. Rotello, \emph{Chem. Soc. Rev.}, 2002, \textbf{31},
  275--286\relax
\mciteBstWouldAddEndPuncttrue
\mciteSetBstMidEndSepPunct{\mcitedefaultmidpunct}
{\mcitedefaultendpunct}{\mcitedefaultseppunct}\relax
\EndOfBibitem
\bibitem[Altintas \emph{et~al.}(2015)Altintas, Artar, ter Huurne, Voets,
  Palmans, Barner-Kowollik, and Meijer]{doi:10.1021/acs.macromol.5b01990}
O.~Altintas, M.~Artar, G.~ter Huurne, I.~K. Voets, A.~R.~A. Palmans,
  C.~Barner-Kowollik and E.~W. Meijer, \emph{Macromolecules}, 2015,
  \textbf{48}, 8921--8932\relax
\mciteBstWouldAddEndPuncttrue
\mciteSetBstMidEndSepPunct{\mcitedefaultmidpunct}
{\mcitedefaultendpunct}{\mcitedefaultseppunct}\relax
\EndOfBibitem
\bibitem[Ji \emph{et~al.}(2015)Ji, Jie, Zimmerman, and Huang]{C4PY01715C}
X.~Ji, K.~Jie, S.~C. Zimmerman and F.~Huang, \emph{Polym. Chem.}, 2015,
  \textbf{6}, 1912--1917\relax
\mciteBstWouldAddEndPuncttrue
\mciteSetBstMidEndSepPunct{\mcitedefaultmidpunct}
{\mcitedefaultendpunct}{\mcitedefaultseppunct}\relax
\EndOfBibitem
\bibitem[Hutchins \emph{et~al.}(2015)Hutchins, Groeneman, Reinheimer, Swenson,
  and MacGillivray]{C5SC00988J}
K.~M. Hutchins, R.~H. Groeneman, E.~W. Reinheimer, D.~C. Swenson and L.~R.
  MacGillivray, \emph{Chem. Sci.}, 2015, \textbf{6}, 4717--4722\relax
\mciteBstWouldAddEndPuncttrue
\mciteSetBstMidEndSepPunct{\mcitedefaultmidpunct}
{\mcitedefaultendpunct}{\mcitedefaultseppunct}\relax
\EndOfBibitem
\bibitem[Handke \emph{et~al.}(2017)Handke, Adachi, Hu, and
  Ward]{https://doi.org/10.1002/anie.201707097}
M.~Handke, T.~Adachi, C.~Hu and M.~D. Ward, \emph{Angew. Chem. Int. Ed.}, 2017,
  \textbf{56}, 14003--14006\relax
\mciteBstWouldAddEndPuncttrue
\mciteSetBstMidEndSepPunct{\mcitedefaultmidpunct}
{\mcitedefaultendpunct}{\mcitedefaultseppunct}\relax
\EndOfBibitem
\bibitem[Bloom and Wheeler(2011)]{Bloom:11}
J.~W.~G. Bloom and S.~E. Wheeler, \emph{Angew. Chem. Int. Ed.}, 2011,
  \textbf{50}, 7847--7849\relax
\mciteBstWouldAddEndPuncttrue
\mciteSetBstMidEndSepPunct{\mcitedefaultmidpunct}
{\mcitedefaultendpunct}{\mcitedefaultseppunct}\relax
\EndOfBibitem
\bibitem[Ahmed \emph{et~al.}(2023)Ahmed, Sundholm, Daub, and
  Johansson]{Usman:paper-1}
U.~Ahmed, D.~Sundholm, C.~Daub and M.~P. Johansson, \emph{to be published},
  2023, \textbf{0}, 0000\relax
\mciteBstWouldAddEndPuncttrue
\mciteSetBstMidEndSepPunct{\mcitedefaultmidpunct}
{\mcitedefaultendpunct}{\mcitedefaultseppunct}\relax
\EndOfBibitem
\bibitem[jmo()]{jmol}
\emph{{{\rm Jmol}: an open-source Java viewer for chemical structures in 3D.}},
  \url{http://www.jmol.org/}~(accessed~15.8.2023)\relax
\mciteBstWouldAddEndPuncttrue
\mciteSetBstMidEndSepPunct{\mcitedefaultmidpunct}
{\mcitedefaultendpunct}{\mcitedefaultseppunct}\relax
\EndOfBibitem
\bibitem[Furche \emph{et~al.}(2014)Furche, Ahlrichs, H\"{a}ttig, Klopper,
  Sierka, and Weigend]{https://doi.org/10.1002/wcms.1162}
F.~Furche, R.~Ahlrichs, C.~H\"{a}ttig, W.~Klopper, M.~Sierka and F.~Weigend,
  \emph{WIREs Comput. Mol. Sci.}, 2014, \textbf{4}, 91--100\relax
\mciteBstWouldAddEndPuncttrue
\mciteSetBstMidEndSepPunct{\mcitedefaultmidpunct}
{\mcitedefaultendpunct}{\mcitedefaultseppunct}\relax
\EndOfBibitem
\bibitem[Staroverov \emph{et~al.}(2003)Staroverov, Scuseria, Tao, and
  Perdew]{doi:10.1063/1.1626543}
V.~N. Staroverov, G.~E. Scuseria, J.~Tao and J.~P. Perdew, \emph{J. Chem.
  Phys.}, 2003, \textbf{119}, 12129--12137\relax
\mciteBstWouldAddEndPuncttrue
\mciteSetBstMidEndSepPunct{\mcitedefaultmidpunct}
{\mcitedefaultendpunct}{\mcitedefaultseppunct}\relax
\EndOfBibitem
\bibitem[Staroverov \emph{et~al.}(2004)Staroverov, Scuseria, Tao, and
  Perdew]{doi:10.1063/1.1795692}
V.~N. Staroverov, G.~E. Scuseria, J.~Tao and J.~P. Perdew, \emph{J. Chem.
  Phys.}, 2004, \textbf{121}, 11507--11507\relax
\mciteBstWouldAddEndPuncttrue
\mciteSetBstMidEndSepPunct{\mcitedefaultmidpunct}
{\mcitedefaultendpunct}{\mcitedefaultseppunct}\relax
\EndOfBibitem
\bibitem[Weigend(2006)]{B515623H}
F.~Weigend, \emph{Phys. Chem. Chem. Phys.}, 2006, \textbf{8}, 1057--1065\relax
\mciteBstWouldAddEndPuncttrue
\mciteSetBstMidEndSepPunct{\mcitedefaultmidpunct}
{\mcitedefaultendpunct}{\mcitedefaultseppunct}\relax
\EndOfBibitem
\bibitem[Weigend and Ahlrichs(2005)]{B508541A}
F.~Weigend and R.~Ahlrichs, \emph{Phys. Chem. Chem. Phys.}, 2005, \textbf{7},
  3297--3305\relax
\mciteBstWouldAddEndPuncttrue
\mciteSetBstMidEndSepPunct{\mcitedefaultmidpunct}
{\mcitedefaultendpunct}{\mcitedefaultseppunct}\relax
\EndOfBibitem
\bibitem[Grimme \emph{et~al.}(2010)Grimme, Antony, Ehrlich, and
  Krieg]{doi:10.1063/1.3382344}
S.~Grimme, J.~Antony, S.~Ehrlich and H.~Krieg, \emph{J. Chem. Phys.}, 2010,
  \textbf{132}, 154104\relax
\mciteBstWouldAddEndPuncttrue
\mciteSetBstMidEndSepPunct{\mcitedefaultmidpunct}
{\mcitedefaultendpunct}{\mcitedefaultseppunct}\relax
\EndOfBibitem
\bibitem[Grimme \emph{et~al.}(2011)Grimme, Ehrlich, and
  Goerigk]{https://doi.org/10.1002/jcc.21759}
S.~Grimme, S.~Ehrlich and L.~Goerigk, \emph{J. Comput. Chem.}, 2011,
  \textbf{32}, 1456--1465\relax
\mciteBstWouldAddEndPuncttrue
\mciteSetBstMidEndSepPunct{\mcitedefaultmidpunct}
{\mcitedefaultendpunct}{\mcitedefaultseppunct}\relax
\EndOfBibitem
\bibitem[{Dunning Jr}(1989)]{Dunning:89}
T.~H. {Dunning Jr}, \emph{J. Chem. Phys.}, 1989, \textbf{90}, 1007--1023\relax
\mciteBstWouldAddEndPuncttrue
\mciteSetBstMidEndSepPunct{\mcitedefaultmidpunct}
{\mcitedefaultendpunct}{\mcitedefaultseppunct}\relax
\EndOfBibitem
\bibitem[Dunning(1989)]{doi:10.1063/1.456153}
T.~H. Dunning, \emph{J. Chem. Phys.}, 1989, \textbf{90}, 1007--1023\relax
\mciteBstWouldAddEndPuncttrue
\mciteSetBstMidEndSepPunct{\mcitedefaultmidpunct}
{\mcitedefaultendpunct}{\mcitedefaultseppunct}\relax
\EndOfBibitem
\bibitem[Davidson(1996)]{DAVIDSON1996514}
E.~R. Davidson, \emph{Chem. Phys. Lett.}, 1996, \textbf{260}, 514--518\relax
\mciteBstWouldAddEndPuncttrue
\mciteSetBstMidEndSepPunct{\mcitedefaultmidpunct}
{\mcitedefaultendpunct}{\mcitedefaultseppunct}\relax
\EndOfBibitem
\bibitem[Weigend \emph{et~al.}(2003)Weigend, Furche, and
  Ahlrichs]{doi:10.1063/1.1627293}
F.~Weigend, F.~Furche and R.~Ahlrichs, \emph{J. Chem. Phys.}, 2003,
  \textbf{119}, 12753--12762\relax
\mciteBstWouldAddEndPuncttrue
\mciteSetBstMidEndSepPunct{\mcitedefaultmidpunct}
{\mcitedefaultendpunct}{\mcitedefaultseppunct}\relax
\EndOfBibitem
\bibitem[Gonthier \emph{et~al.}(2012)Gonthier, Steinmann, Roch, Ruggi, Luisier,
  Severin, and Corminboeuf]{C2CC33886F}
J.~F. Gonthier, S.~N. Steinmann, L.~Roch, A.~Ruggi, N.~Luisier, K.~Severin and
  C.~Corminboeuf, \emph{Chem. Comm.}, 2012, \textbf{48}, 9239--9241\relax
\mciteBstWouldAddEndPuncttrue
\mciteSetBstMidEndSepPunct{\mcitedefaultmidpunct}
{\mcitedefaultendpunct}{\mcitedefaultseppunct}\relax
\EndOfBibitem
\bibitem[Lu and Chen(2012)]{https://doi.org/10.1002/jcc.22885}
T.~Lu and F.~Chen, \emph{J. Comput. Chem.}, 2012, \textbf{33}, 580--592\relax
\mciteBstWouldAddEndPuncttrue
\mciteSetBstMidEndSepPunct{\mcitedefaultmidpunct}
{\mcitedefaultendpunct}{\mcitedefaultseppunct}\relax
\EndOfBibitem
\bibitem[{Gijs Schaftenaar}()]{molden}
{Gijs Schaftenaar}, \emph{Molden}, \url{https://www.theochem.ru.nl/molden/}
  (accessed 18.8.2023)\relax
\mciteBstWouldAddEndPuncttrue
\mciteSetBstMidEndSepPunct{\mcitedefaultmidpunct}
{\mcitedefaultendpunct}{\mcitedefaultseppunct}\relax
\EndOfBibitem
\bibitem[Neese(2012)]{https://doi.org/10.1002/wcms.81}
F.~Neese, \emph{WIREs Comput. Mol. Sci.}, 2012, \textbf{2}, 73--78\relax
\mciteBstWouldAddEndPuncttrue
\mciteSetBstMidEndSepPunct{\mcitedefaultmidpunct}
{\mcitedefaultendpunct}{\mcitedefaultseppunct}\relax
\EndOfBibitem
\bibitem[Neese(2018)]{https://doi.org/10.1002/wcms.1327}
F.~Neese, \emph{WIREs Comput. Mol. Sci.}, 2018, \textbf{8}, e1327\relax
\mciteBstWouldAddEndPuncttrue
\mciteSetBstMidEndSepPunct{\mcitedefaultmidpunct}
{\mcitedefaultendpunct}{\mcitedefaultseppunct}\relax
\EndOfBibitem
\bibitem[Sch{\"a}fer \emph{et~al.}(1992)Sch{\"a}fer, Horn, and
  Ahlrichs]{Schafer:92}
A.~Sch{\"a}fer, H.~Horn and R.~Ahlrichs, \emph{J. Chem. Phys.}, 1992,
  \textbf{97}, 2571--2577\relax
\mciteBstWouldAddEndPuncttrue
\mciteSetBstMidEndSepPunct{\mcitedefaultmidpunct}
{\mcitedefaultendpunct}{\mcitedefaultseppunct}\relax
\EndOfBibitem
\bibitem[Steinmann \emph{et~al.}(2011)Steinmann, Mo, and
  Corminboeuf]{C1CP21055F}
S.~N. Steinmann, Y.~Mo and C.~Corminboeuf, \emph{Phys. Chem. Chem. Phys.},
  2011, \textbf{13}, 20584--20592\relax
\mciteBstWouldAddEndPuncttrue
\mciteSetBstMidEndSepPunct{\mcitedefaultmidpunct}
{\mcitedefaultendpunct}{\mcitedefaultseppunct}\relax
\EndOfBibitem
\bibitem[Williams \emph{et~al.}(2013)Williams, Kelley, and
  {co-workers}]{gnuplot}
T.~Williams, C.~Kelley and {co-workers}, \emph{Gnuplot 4.6: an interactive
  plotting program},
  \url{http://gnuplot.sourceforge.net/}~(accessed~15.8.2023), 2013\relax
\mciteBstWouldAddEndPuncttrue
\mciteSetBstMidEndSepPunct{\mcitedefaultmidpunct}
{\mcitedefaultendpunct}{\mcitedefaultseppunct}\relax
\EndOfBibitem
\end{mcitethebibliography}
\end{document}
